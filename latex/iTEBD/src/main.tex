\documentclass{uofa-eng-assignment}
\usepackage{float}
\usepackage{bm}
\usepackage{ctex}
\usepackage{xcolor}
\usepackage{subfigure}
\usepackage{tikz}
\usepackage{mathrsfs}
\usepackage{subfig}
\usepackage{physics}
\newcommand*{\name}{universe9793}
\newcommand*{\id}{320200910461}
\newcommand*{\course}{线性代数\uppercase\expandafter{\romannumeral2}}
\newcommand*{\assignment}{入门张量网络方法}
%\renewcommand{\familydefa\mathrm{U}lt}{\ttdefa\mathrm{U}lt}

%\setCJKmainfont{思源黑体 CN Normal}
\begin{document}
\maketitle
\begin{center}
    \textbf{\Large 摘要}
\end{center}
    \qquad 这篇文章主要记录了入门学习张量网络的过程,主要的过程为:首先学习张量的一些基本知识,包括张量的矩阵化、张量的缩并、
    张量的图形表示; 接下来学习了一些张量的分解过程,Tucker分解以及HOSVD分解,了解了一下什么是最优低秩近似, 并且在python
    上写了一些简单代码进行了测试; 之后复习了一下Power Method,学习T-S分解;最后便是学习Matrix Product State,包括怎么
    通过SVD分解构建MPS态,怎么对MPS进行正则化。在最后就是解决一些实际的问题:使用iTEBD(虚时演化,本质是Power Method)估算
    一维无限长自旋1/2(1)反铁磁海森堡链的基态能量,学习怎么计算算符的期望值;求解二维方格子无限大自旋1/2(1)反铁磁海森堡模型
    的基态能量;求解Kagome晶格无限大自旋1/2(1)反铁磁海森堡模型的基态能量。

\section[Tensor]{张量基本知识}\label{sec:Tensor}
    这一节主要记录学习到的张量基本知识。
    \subsection[Tenser Size]{张量的一些概念}\label{subsec:Size}
        \textbf{张量的概念}:最简单来讲,张量就是由多个指标所标记的一系列数;

        \textbf{张量的阶数}:指标的个数;

        \textbf{指标的维数}:每一个指标所能取到的值的个数;
    
        \textbf{张量元}:构成张量的数.
    \subsection[Tensor Operator]{张量的基本操作}\label{subsec:operator}
        常见的张量的操作有切片、变形、转置。

        \textbf{切片}:




\end{document}
